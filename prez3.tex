\documentclass{beamer}
\usepackage{graphicx}
\usepackage[croatian]{babel}
\usepackage{csquotes}
\usepackage{amsfonts, amsmath, amsthm}
\usepackage{mathtools}
\usepackage{thmtools}
%\usepackage[backend=biber,style=authoryear,autocite=inline]{biblatex}
\usepackage[backend=biber]{biblatex}
\addbibresource{seminar.bib}
%\bibstyle{alpha}
%\bibliographystyle{alpha}
\MakeOuterQuote{"}

\def\N{\mathbb{N}}
\def\Z{\mathbb{Z}}
\def\Q{\mathbb{Q}}
\def\R{\mathbb{R}}
\def\C{\mathbb{C}}
\def\CR{\mathrm{CR}}

%Information to be included in the title page:
\title{Generalizirani kalkulator}
\subtitle{u Haskellu}
\author{Mauro Raguzin}
\institute{PMF-MO}
\date{\today}

\begin{document}

\frame{\titlepage}

\begin{frame}
    \frametitle{Sadržaj}
    \tableofcontents
\end{frame}

\section{Uvodno o projektu}
% \begin{frame}
% \frametitle{O kalkulatorima}
% Kalkulatori su nam poznati iz svakodnevice:
% \begin{itemize}
%     \item pružaju brzi način dobivanja rezultata najčešćih računskih operacija
%     \item sadrže rutine za računanje poznatih konstanti poput $\pi$, $e$ itd.
%     \item neki mogu čak crtati i grafove, rješavati jednadžbe\ldots
%     \item sve račune koji uključuju realne brojeve rade do na određenu preciznost
% \end{itemize}
% Iako su nam kalkulatori korisni, bili oni softverski ili hardverski (džepni),
% postavljaju se pitanja:
% \begin{itemize}
%  \item Postoji li garancija da računaju na određenu preciznost?
% \item Ako je preciznost fiksirana, je li ona određena brojem decimala
% ili nekako drukčije?
% \item Možemo li proširiti računanje s osnovnih operacija na proizvoljne
% (očekivano totalne) funkcije nad \alert{realnim} brojevima?
% \end{itemize}
% \end{frame}

\begin{frame}
\frametitle{O ideji ovog kalkulatora}
Neki od ciljeva ovog projekta su bili:
\begin{itemize}
    \item primijeniti razne rezultate iz kolegija Izračunljiva analiza,
    \item konstruirati kalkulator u Haskellu,
    \item proširiti standardne funkcionalnosti onima koje omogućuju
    računanje s proizvoljnim Haskellovim funkcijama nad brojevima.
\end{itemize}
\pause
Kako se često u uvodnim tekstovima o Haskellu zna napomenuti da
se interaktivna GHC sesija može koristiti kao kalkulator, ovdje to
zaista želimo ojačati i proširiti na funkcije --- nije moguće računati
samo s ugrađenim Haskellovim brojevnim tipovima, već i s funkcijama
nad tim tipovima!
\end{frame}

\begin{frame}
\frametitle{Bitne funkcionalnosti kalkulatora}
Konkretno, naš kalkulator podržava:
\begin{itemize}
    \item računanje (obavljanje svih standardnih aritmetičkih
    operacija 
    $+$,$-$,$*$,$/$,$|\cdot|$, no ovisi o konkretnoj algebarskoj
    strukturi) s prirodnim, cijelim, racionalnim i 
    \emph{izračunljivim}\footnote{Skup izračunljivih realnih brojeva
    označavamo sa $\CR\subseteq\R$.} realnim brojevima;
    \item sve aritmetičke operacije nad \alert{funkcijama} oblika
    $\N^k\to D$, gdje je $D\in\{\N,\Z,\Q,\CR\}$ i $k\in\N_+$.
\end{itemize}
U slučaju rada s (implicitno izračunljivim) realnim brojevima,
    korisniku dajemo kontrolu nad određivanjem preciznosti u smislu
    standardne euklidske metrike nad $\R$ --- preciznost je uvijek
    argument koji se mora specificirati pri računu realnog broja.
\end{frame}

\begin{frame}
    \frametitle{Naputci za implementaciju}
    Pri implementaciji kalkulatora u Haskellu nastojali smo se
    pridržavati sljedećih pravila: 
    \begin{itemize}
        \pause
        \item koristiti samo čisti Haskell sa standardnim Prelude-om;
        \pause
        \item definirati sve bitnije funkcije za osnovni račun s brojevnim funkcijama s Izračunljive analize i pokušati se što više
        držati definicija s tog kolegija;
        \pause
        \item postići što veću razinu formalne sličnosti definicija
        relevantnih
        funkcija u raznim dokazima rezultata iz IA i njihovih
        haskellovskih implementacija;
        \pause
        \item uklopiti definicije funkcija u odgovarajuće algebarske
        strukture;
        \pause
        \item reprezentirati algebarske strukture smislenim type-klasama,
        pri čemu treba težiti instanciranju već postojećih Prelude
        klasa, čime također dobivamo mogućnost korištenja standardnih
        infiksnih operatora u radu s našim brojevnim funkcijama
        (à la stolni kalkulatori) 
    \end{itemize}
\end{frame}

\begin{frame}
    \frametitle{Čime se \emph{nismo} bavili}
    Važno je napomenuti i što ovaj kalkulator definitivno nije.
    Naime, on \alert{nije} kalkulator koji
    \begin{itemize}
        \pause
        \item može izračunati sve realne brojeve; smatra se da takvo
        što nije fizički ni teorijski moguće sa sadašnjim razumijevanjem 
        izračunljivosti \cite{weihrauch};
        \pause
        \item koristi posebno odabrane, efikasne algoritme za 
        svoje računske operacije;
        \pause
        \item može računati korijene polinoma, rješavati sustave jednadžbi\ldots 
        \pause
        \item garantira terminaciju svih svojih izračunavanja.
    \end{itemize}
    \pause
    U vezi s posljednjom točkom, napomenimo da bi za stolni
    tip kalkulatora bilo idealno osigurati terminaciju svih operacija;
    one koje su ilegalne bi se trebale detektirati te bi se korisniku
    ispisala greška. \pause No, kako je naš implementacijski jezik Haskell,
    te želimo nešto općenitiji kalkulator koji radi s korisnički
    isprogramiranim funkcijama te je Haskell Turing-potpun jezik,
    smatramo da nema smisla korisniku davati neke posebne konstrukte
    koje ga ograničavaju u izražajnosti. 
\end{frame}

\begin{frame}
\frametitle{Ograničenja}
\begin{block}{Napomena}
Terminacija izračunavanja
    funkcije u Haskellu, za dani ulaz, dakako nije odlučivo svojstvo
    i zato bi jedini način forsiranja terminacije bio uvođenje posebnog
    pod-jezika koji ne bi mogao biti Turing-potpun, što bi nas vratilo
    na razinu primitivnih kalkulatora koju želimo nadići.
\end{block}

\pause
Dakle, dozvoljavamo da korisnik završi s parcijalnom funkcijom prilikom
čijeg izračunavanja na našem kalkulatoru može doći do divergencije
izračunavanja, što u Haskellu znači da funkcija vraća tip $\bot$.

\pause
\begin{alertblock}{Pažnja}
Kako se na IA gotovo uvijek pretpostavlja rad
s \alert{rekurzivnim} funkcijama, što bi u Haskellu odgovaralo totalnim
funkcijama --- onima koje ne mogu divergirati --- tvrdnje s tog
kolegija u našem kalkulatoru vrijede samo ako korisnik
zna da operira nad funkcijama
koje su same totalne.
\end{alertblock}
\end{frame}

\begin{frame}
    \frametitle{Ograničenja (nastavak)}
    Kako totalnost funkcije nije općenito moguće odlučiti u bilo
    kojem programskom jeziku, moramo se pouzdati u korisnika ili u njegovo
    korištenje posebnih konstrukcija u svom kodu kojima garantira da sve
    tako dobivene funkcije jesu totalne.    
\end{frame}

\begin{frame}
    \frametitle{Osnovno o izračunljivosti brojevnih funkcija}
    Na kolegiju Izračunljivost smo primarno promatrali izračunljivost
    u okviru modela parcijalno rekurzivnih funkcija oblika
    \begin{equation*}
        \N^k\to\N,
    \end{equation*}
    za pozitivnu mjesnost $k$. Na to smo se  oslonili i na
     Izračunljivoj analizi, s tim da smo tamo dali formalna proširenja
     tog modela izračunljivosti na funkcije oblika
     \begin{equation*}
        \N^k\to\Z,\qquad \N^k\to\Q,\qquad \N^k\to\R,
     \end{equation*}
    u pravilu pretpostavljajući u rezultatima da radimo s takvim
    funkcijama koje su ujedno i rekurzivne tj. totalne i izračunljive. 
\end{frame}

\begin{frame}
    \frametitle{U čemu je problem?}
    \begin{itemize}
        \item U Haskellu, kao ni u praktički bilo kojem programskom jeziku,
        nije pretjerano teško implementirati računanje s funkcijama
        svih spomenutih oblika, $\N^k\to\R$
\item 
    Taj zadnji oblik
     zahtijeva malo drukčiji pristup, budući da on, za razliku od
     svih ostalih navedenih, obuhvaća funkcije koje preslikavaju
     prirodne brojeve --- koji se mogu konačno reprezentirati na
     računalu --- u realne brojeve, koji se tako ne mogu reprezentirati
\item Ipak, to nas ne spriječava da računamo \alert{proizvoljno točne
aproksimacije} odgovarajućih realnih brojeva --- izračunljive
Cauchyjeve nizove
(racionalnih brojeva) koji jako brzo konvergiraju ka realnim brojevima.
    \end{itemize}
\end{frame}

\begin{frame}
    \frametitle{Izračunljivi realni brojevi $\CR$}
    Tako dolazimo do centralne definicije iz Izračunljive analize.
    \begin{block}{Definicija}
    Realan broj $\alpha\in\R$ nazivamo izračunljivim realnim brojem
    ako postoji rekurzivna funkcija $F:\N\to\Q$ takva da vrijedi
    \begin{equation*}
        |\alpha-F(k)| < 2^{-k},
    \end{equation*}
    za sve $k\in\N$.
    \end{block}
    $k$ koristimo kao garantiranu preciznost aproksimacije koju dobivamo
    preko rekurzivne (u Haskellu, totalne) funkcije $F$. Na ovoj se
    definiciji, i mnogim rezultatima iz IA, temelji naša implementacija
    izračunavanja s (izračunljivim) realnim brojevima.
\end{frame}

\begin{frame}
    \frametitle{O prezentaciji kalkulatora}
    \begin{itemize}
        \item U nastavku ćemo proći kroz sve rezultate iz IA
        koji su relevantni
        za ovaj projekt
        \item Usporednim prikazom slike originalnih
        materijala iz IA i slike izvornog koda kalkulatora ćemo vidjeti
        kako su određeni matematički konstrukti implementirani u Haskellu
        \item Definicije tipova $\Z$ i $\Q$ u Haskellu nisu dane, jer se
        koriste posebno optimizirane interne GHC-ove implementacije tih tipova
        (\texttt{Integer} odnosno \texttt{Rational})
        \item Iako ne otkrivamo ništa novo pokazivanjem kako se računa
        s funkcijama oblika, $\N^k\to\Z$ i $\N^k\to\Q$, ispada da
        se implementacija operacija nad takvim funkcijama može definirati
        na sasvim uniforman način, čak i kad se uzmu u obzir i funkcije
        oblika $\N^k\to\CR$.    
    \end{itemize}
\end{frame}


\section{Implementacija}
\begin{frame}
    \frametitle{Prirodni brojevi}
    Prvo i osnovno, treba nam tip koji reprezentira $\N$. Haskell
    ga nema definiranog u Prelude-u, pa ćemo napraviti jednostavnu
    rekurzivnu definiciju i instancirati sve klase čija svojstva
    $\N$ zadovoljava. Primijetimo da ćemo pritom ipak dozvoliti da
    tip \texttt{Nat} ima funkciju \texttt{negate} za negaciju, iako
    ta operacija nema smisla, jer želimo iskoristiti Haskellovu
    predefiniranu \texttt{Num} klasu koja nam omogućuje korištenje
    standardnih infiksnih operatora za aritmetičke operacije.
    Jedini operator koji ovdje nije definiran je onaj za inverz
    tj. recipročnu vrijednost broja \texttt{recip}; on se nalazi
    u klasi \texttt{Fractional} koja modelira tipove brojeva koji
    zadovoljavaju aksiome polja. To dakle ostavljamo za $\Q$ i $\CR$.
\end{frame}    

\begin{frame}
\frametitle{Definicija \texttt{Nat} i pripadnih instanci}
\includegraphics{natdef.PNG}
\end{frame}
\begin{frame}
\includegraphics[scale=0.7]{natclass1.PNG}
\end{frame}
\begin{frame}
\includegraphics[scale=0.7]{natclass2.PNG}
\end{frame}
\begin{frame}
\begin{columns}
    \begin{column}{0.5\textwidth}
    \begin{equation*}
        |x-y|=(x\dot{-}y)+(y\dot{-}x),\,x,y\in\N
    \end{equation*}
    \includegraphics[scale=0.5]{natclass3.PNG}
\end{column}
\begin{column}{0.6\textwidth}
    \begin{align*}
        \qquad\mathrm{sg}(n)&=\begin{cases}
            0,&\text{ako je $n=0$;}\\
            1,&\text{inače.}
        \end{cases} \\
        \overline{\mathrm{sg}}(n)&=\begin{cases}
            1,&\text{ako je $n=0$;}\\
            0,&\text{inače.}
        \end{cases}
    \end{align*}
\end{column}
\end{columns}
\end{frame}
\begin{frame}
    \includegraphics[scale=1]{funmain.PNG}
\end{frame}
\begin{frame}
    \includegraphics[scale=1]{iaring.PNG}
    \includegraphics[scale=0.65]{ringdefs.PNG}
\end{frame}
\begin{frame}
    \includegraphics[scale=1]{iaring.PNG}
    \includegraphics[scale=0.62]{ringimpl1.PNG}
\end{frame}
\begin{frame}
\includegraphics[scale=1]{iafunops.PNG}
\includegraphics[scale=1]{funopdefs.PNG}%NAPOMENI: konstantne su funkcije
%uvijek rekurzivne, to je trivijalno i zato uvijek imamo fromInteger i sl. funkcije
\end{frame}
\begin{frame}
    \includegraphics[scale=0.8]{funopext.PNG}
\end{frame}
\begin{frame}
    \includegraphics[scale=0.42]{realring.png}
    Uočavamo "univerzalnu konstrukciju":
    \begin{equation*}
        (\mathrm{Fun1}\ f) \oplus (\mathrm{Fun1}\ g) = \mathrm{Fun1} (\lambda x\rightarrow f x \oplus g x)
    \end{equation*}
\end{frame}
\begin{frame}
    Moramo implementirati vlastitu kompoziciju i aplikaciju (poziv) funkcija, za sve različite kombinacije mjesnosti. Iz IA
    znamo da je kompozicija rekurzivnih funkcija (svih koje smo spomenuli) rekurzivna funkcija.
    \includegraphics[scale=0.5]{compose.png}
\end{frame}
\begin{frame}
    Još treba implementirati \texttt{Fractional} --- to su svi tipovi brojeva koji imaju multiplikativni inverz,
    te sada želimo proširiti računanje njima na funkcije s tom kodomenom.
    \includegraphics[scale=0.4]{fractionalfun.png}
    \includegraphics[scale=0.3]{fractional_korolar.png}
\end{frame}
\begin{frame}
    \frametitle{Izračunljivi realni brojevi}
    Sada dolazi zanimljiv dio: računanje s (izračunljivim) realnim brojevima. Prvo moramo definirati taj tip broja.
    Trebaju nam i pomoćne funkcije za aproksimiranje te kombinirano apliciranje i aproksimiranje u slučaju
    \emph{funkcija} s kodomenom $\CR$.
    \includegraphics[scale=0.5]{realdef.png}
\end{frame}
\begin{frame}
    \includegraphics[scale=0.5]{realfuns.png}
    Uočavamo još jednostavniju konstrukciju (uz iznimku) svih operacija nad izračunljivim realnim brojevima:
    \begin{equation*}
        (\CR\ x) \oplus (\CR\ y) = \CR(x\oplus y)
    \end{equation*}
\end{frame}
\begin{frame}
    Odgovarajući dokazi iz Izračunljive analize za produkt i recipročnu vrijednost.
    % \begin{columns}
    %     \begin{column}{0.5\textwidth}

    % \end{columns}
    \includegraphics[scale=0.15]{produkt.png}
    \mbox{\ldots}\vspace{4pt}
    %\hline
    \includegraphics[scale=0.2]{recip1.png}
    \includegraphics[scale=0.2]{recip2.png}
\end{frame}
\begin{frame}
    \frametitle{Digresija: alternativa?}
    \includegraphics[scale=0.26]{digresija.png}
\end{frame}
\begin{frame}
    Ipak moramo posebno definirati funkcije iz naše dodatne klase, jer izračunljivi realni brojevi nemaju odlučiv uređaj, pa ne mogu
    koristiti prošlu implementaciju.
    \includegraphics[scale=0.3]{ringreals.png}
\end{frame}
\begin{frame}
    \includegraphics[scale=0.5]{decimal_ia.png}
    \includegraphics[scale=0.28]{decimal.png}
    \includegraphics[scale=0.4]{decimal_ia2.png}
\end{frame}
\begin{frame}
    Funkcija $u'$ u dokazu:
    \includegraphics[scale=0.6]{ufun.png}
\end{frame}
\begin{frame}
    Za prikaz proizvoljno mnogo decimala pozitivnog izračunljivog broja (uključujući i njegov cjelobrojni dio), koristimo funkciju
    \texttt{showReal}:
    \includegraphics[scale=0.4]{showreal.png}
\end{frame}
\begin{frame}
    Na Izračunljivoj analizi smo istaknuli da možemo "promovirati" funkcije stupnjevito, pa ovdje to eksplicitno 
    definiramo:
    \includegraphics[scale=0.4]{promocija.png}
\end{frame}

\begin{frame}
    \frametitle{Račun s funkcijama oblika $f:\subseteq\R\to\R$}
    \begin{itemize}
        \item Htjeli bismo još da naš kalkulator može računati i s funkcijama koje kao domenu imaju neki podskup od $\R$.
        \pause
        \item Postoji više smislenih načina za definirati izračunljivost nad ovakvim funkcijama.
        \pause
        \item Definicija s IA je malo prespecifična za naše potrebe, pa mi uvodimo sljedeću definiciju:
    \end{itemize}
    \begin{block}{Definicija}
        Za funkciju $f:S\to\CR,\,S\subseteq\R$ kažemo da je izračunljiva ako za svaki $x\in S$ vrijedi da je
        $f(x)$ izračunljiv realan broj.
    \end{block}
    Možemo dakle reći i ovako: $f:S\to\CR$ je izračunljiva ako je $S\subseteq\CR$.
\end{frame}
\begin{frame}
    \frametitle{Računanje polinoma}
    \begin{align*}
        p:\R&\to\R \\
        p(x)=a_0+a_1x&+\dotsb+a_nx^n
    \end{align*}
    Polinome lako računamo direktno i zadajemo ih listom koeficijenata, gdje je prvi slobodan:
    \includegraphics[scale=0.6]{poly.png}
\end{frame}
\begin{frame}
    \frametitle{Računanje redova potencija}
    Neka je zadan rekurzivan (izračunljiv) niz realnih brojeva $(a_n)_{\N}$. Onda za neki $R\in\langle 0,\infty\rangle\cup\{\infty\}$ imamo
    \begin{equation*}
        \sum^{\infty}_{i=0}a_ix^i,\,x\in\langle -R,R\rangle.
    \end{equation*}
    \pause
    Račun ide nešto teže --- zahtijeva mnoštvo dodatnih rezultata. Naime, pokazuje se da je račun reda općenito moguć samo ako se uz argument $x$ u području
    konvergencije zadaju i tri dodatna parametra takva da vrijedi odnos: $|x|<s<r$ te $M\in\N$, $r,s\in\Q$.
    \pause
    
    Iz analize znamo da za svaki $r<R$ postoji konstanta $M\in\N$ takva da vrijedi
    \begin{equation*}
        |a_j|\leq M\cdot r^{-j}\text{ za sve }j\in\N.
    \end{equation*}\cite{weihrauch}
\end{frame}
\begin{frame}
    \includegraphics[scale=0.35]{series.png}
\end{frame}

\section{Zaključak}
\begin{frame}
    \frametitle{Moguća poboljšanja/proširenja}
    \begin{itemize}
        \item odabir najefikasnijih algoritama i notacija/reprezentacija za računanje s realnim brojevima; mnoge dobre ideje npr. u \cite{ram};
        \item omogućavanje računanja uređaja, s tim da on postaje \alert{ograničene preciznosti} --- unutar određene kritične zone ne može biti 
        determinističan;
        \item dizajn zasebnog programskog jezika specijaliziranog za računanje s tipom $\CR$, funkcijama nad njim i moguće još nekim vezanim strukturama;
        \item parametrizacija mjesnosti u Haskellu (ne pomoću nekog metajezika poput Template Haskella!), koristeći neke GHC ekstenzije;
        \item dodati rješavač linearnih sustava, vađenje korijena polinoma\ldots
        \item implicitno/automatsko procjenjivanje aritmetičke složenosti cjelokupnog programa, ovisno o statički specificiranim parametrima
        preciznosti na ključnim mjestima
    \end{itemize}
\end{frame}


\begin{frame}
    \frametitle{Literatura}
    \printbibliography
\end{frame}

\end{document}